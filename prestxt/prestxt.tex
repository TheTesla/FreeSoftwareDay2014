
\documentclass[xetex]{beamer}

\usepackage[ngerman]{babel}
\usepackage{graphicx}
\usepackage{pifont}
\usepackage{fontspec}
\setmainfont{unifont}

\mode<presentation> {
\usetheme{Frankfurt}
\setbeamertemplate{navigation symbols}{}
}

\newcommand{\frameofframes}{/}
\newcommand{\setframeofframes}[1]{\renewcommand{\frameofframes}{#1}}

\setframeofframes{of}
\makeatletter
\setbeamertemplate{footline}
  {%
    \begin{beamercolorbox}[colsep=1.5pt]{upper separation line foot}
    \end{beamercolorbox}
    \begin{beamercolorbox}[ht=2.5ex,dp=1.125ex,%
      leftskip=.3cm,rightskip=.3cm plus1fil]{author in head/foot}%
      \leavevmode{\usebeamerfont{author in head/foot}\insertshortauthor}%
      \hfill%
      {\usebeamerfont{institute in head/foot}\usebeamercolor[fg]{institute in head/foot}\insertshortinstitute}%
    \end{beamercolorbox}%
    \begin{beamercolorbox}[ht=2.5ex,dp=1.125ex,%
      leftskip=.3cm,rightskip=.3cm plus1fil]{title in head/foot}%
      {\usebeamerfont{title in head/foot}\insertshorttitle}%
      \hfill%
      {\usebeamerfont{frame number}\usebeamercolor[fg]{frame number}\insertframenumber~\frameofframes~\inserttotalframenumber}
    \end{beamercolorbox}%
    \begin{beamercolorbox}[colsep=1.5pt]{lower separation line foot}
    \end{beamercolorbox}
  }
\makeatother

\title[I \ding{170} Free Softwarfe -- \insertsubtitle]{I \ding{170} Free Software}
\subtitle{Warum freie Software?}
\author{Matthias Fritzsche, Stefan Helmert}
\institute[ChaosChemnitz]{ChaosTreff Chemnitz
\medskip \textit{info@chaoschemnitz.de}}
\date{14. Februar 2014}

\begin{document}

\begingroup
\renewcommand*\dohead{\rule{0em}{1.49em}}
\frame{\titlepage} % Print the title page as the first slide



\begin{frame}
\frametitle{Gliederung} % Table of contents slide, comment this block out to remove it
\tableofcontents % Throughout your presentation, if you choose to use \section{} and \subsection{} commands, these will automatically be printed on this slide as an overview of your presentation
\end{frame}
\endgroup

\begingroup
\section{Vorteile!?}
\begin{frame}
\frametitle{Vorteile!?}
\begin{itemize}
\item Wartbarkeit/Anpassbarkeit
\item Testbarkeit -- Compilerunabh�ngig
\item Fehler k�nnen von unabh�ngigen Personen gefunden werden
\begin{itemize}
    \item Wer hat schon Bock darauf das Kauderwelsch von Narzisten zu �berpr�fen, die nicht in der Lage sind mit dieser "`Arbeit"' Geld zu verdienen?
\end{itemize}
\item gr��erer Druck auf die Entwickler, bei Fehlern -- N�, ist doch kostenloser Ramsch
\item daher meist zeitnahe Behebung -- Firefox-Versionsnummer als Beweis
\item bei GPL: Zwang �nderungen/Erweiterungen zu ver�ffentlichen
\end{itemize}
\end{frame}


\begin{frame}
\frametitle{noch mehr Vorteile!}
\begin{itemize}
    \item Top Doku -- weil jeder auf der Welt die Software kostenlos ausprobieren und penibel dokumentieren kann, auch den Quellcode -- nat�rlich �bersichtlich in Doxygen
    \item Top Support -- begeisterte Nutzer und Entwickler wollen kein Geld verdienen -- sie wollen Ruhm und Ehre, indem sie alle durch selbstlose Unterst�tzung von den Vorteilen ihrer Software �berzeugen
    \item 100 \% zuverl�ssig und fehlerfrei -- ausgiebige Test durch viele unabh�ngige Tester -- Testen ist bei kommerzieller Software extrem teuer, d. h. nicht wirtschaftlich m�glich
    \item schnelle Reaktion auf den Markt -- bevor kapitalistische CS-Nazis auf den Markt dr�ngen
    \item Riesige Auswahl -- Entwicklung ist ja schlie�lich kostenlos
    \item Es gibt keine Sicherheitsl�cken! Wer den Quellcode durchgelesen h�tte, h�tte erkennen m�ssen, dass die Software nie f�r diesen Zweck gedacht war.
\end{itemize}
\end{frame}

\begin{frame}
\frametitle{die wichtigsten Tatsachen!}
\begin{itemize}
    \item Linux st�rzt nie ab!
    \item Linux hat keine Sicherheitsl�cken!
    \item Linux ist das stabilste Betriebssystem!
    \item Linux ist fehlerfrei!
    \item Unter Linux gibt es Prinzipbedingt niemals Viren.
    \item der GCC enth�lt keine Fehler, insbesondere keine, die die Sicherheit erzeugter Programme beeintr�chtigen k�nnten!
    \item Ohne den Quellecode zu kennen ist es absolut nicht m�glich Software hinschitlich Sicherheitsl�cken oder anderer Schwachstellen zu untersuchen!
    \item Nur OS Verschl�sselungssoftware taugt leider nichts, jeder kann ja sehen wie sie funktioniert.
    \item Die deutsche Automobilindustrie setzt auf frei Software!
\end{itemize}
\end{frame}

\begin{frame}
\frametitle{kaum zu Glauben!}
\begin{itemize}
    \item Die Industrie strebt ein hohes Ma� an Markttransparenz und Wettbewerb an. Sie entwickelt daher modulare freie Software, die von den Wettbewerbern problemlos wiederverwendet und angepasst werden kann! 
\end{itemize}
\end{frame}

\section{mal im Ernst}
\begin{frame}
\frametitle{mal im Ernst}
\textbf{Warum sollte jemand freie Software herstellen?}
\end{frame}

\begin{frame}
\frametitle{mal im Ernst}
\begin{itemize}
    \item weil man mit Hardware Geld verdienen kann
    \begin{itemize}
        \item RaspberryPi
        \item 3D-Drucker
        \item Openmoko
    \end{itemize}
    \item weil man mit Service Geld verdienen kann
    \begin{itemize}
        \item Cloud
        \item Anpassung
        \item Schulung
    \end{itemize}
\end{itemize}
\end{frame}

\begin{frame}
\frametitle{mal im Ernst}
\begin{itemize}
    \item weil man Closed Source Addons verkaufen kann
    \item weil mit Zertifizierungen f�r Drittanbieter Geld verdienen kann
\end{itemize}
\end{frame}

\begingroup
\renewcommand*\dohead{\rule{0em}{1.49em}}
\section*{}
\begin{frame}
\frametitle{Ende}
\centering{Fragen\\
und\\
Antworten.\\}
\end{frame}

\endgroup

\end{document} 
